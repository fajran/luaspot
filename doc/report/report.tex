
% TODO:
% - fix image caption numbering

\documentclass[conference]{IEEEtran}

\usepackage{graphics}
\usepackage{graphicx}
\usepackage{pstricks}
\usepackage{epsfig}
\usepackage{pst-grad} % For gradients
\usepackage{pst-plot} % For axes

\begin{document}

\title{Scriptable Sensor Network}

\author{\IEEEauthorblockN{Fajran Iman Rusadi}
\IEEEauthorblockA{Universiteit van Amsterdam\\
Email: frusadi@science.uva.nl}
\and
\IEEEauthorblockN{ZhengZhangzheng}
\IEEEauthorblockA{Universiteit van Amsterdam\\
Email: zzheng@science.uva.nl}}

\maketitle

\begin{abstract}
Abstract 
\end{abstract}



\section{Introduction}

% - sensor network
%   - Sun SPOT
%     -> we are going to use Sun SPOT
% - scripting language
%   -> what can scripting does
%      -> dynamic execution
% - concept of active network

\subsection{Wireless Sensor Network}

Awireless sensor network (WSN) is a wireless network consisting of spatially
distributed autonomousdevices using sensorsto cooperatively monitor physical or
environmental conditions, such as temperature, sound, vibration, pressure,
motion or pollutants, at different locations \cite{wsn} This
resource-constrained network is usually self-configured and data centric, and
always has dynamic topology and a specific application running on it. In our
experiments, we use SunSpots, powerful sensor devices which are easier to
program to make a more flexible sensor network. 

\subsection{Scripting Language}

A scripting languageis a programming languagethat allows some control of a
single or many software applications. Scripts are often interpreted from the
source code or "semi-compiled" to bytecodewhich is interpreted, unlike the
applications they are associated with, which are traditionally compiledto
native machine code for the system on which they run. Scripting languages are
nearly always embedded in the application with which they are associated. With
thischaracteristicof scriptinglanguage, it can support dynamic execution fairly
well. 

\subsection{Concept of Active Network}

Active networkis an architectural framework to allow extension ofnetwork
services by users. It can support application-specific network (layer)
servicesand programs can inject code fragments to decide how their traffic
isprocessed by the network.The nodes of active network are programmed to
perform custom operations on the messages that pass through the node. For
example, a node could be programmed or customized to handle packets on an
individual user basis or to handle multicastpackets differently than other
packets. 

% ====================================================================================

\section{Background}

% - problems:
%   - we want to utilize h/w resource
%   - we want to run different applications
%     - so we have to put the prog on the sensor network
%     - but if we always have to do this manually, we will
%       have difficulties in doing it: 
%       - waste time, human resources, money, etc
%   - so, we want to be able to deploy/install/put the app
%     dynamically (remotely) to all sensor networks
%   - we also want to change the network behaviour programmatically (active network)

At present, most state of the art sensor networksare designed for a single
application. We can bear this approach on the condition that the device in the
sensor network is not so powerful and expensive. But with the advent of more
powerful senor devices, we want to utilize the hardware we invest in the sensor
network to a bigger extent, hoping that it can run multiple applications in it,
not restricted by simply measuring temperatureand humidity. Traditionally, the
administrator of the sensor network can collect all the devices in the network
from different locations, upgrade the firmware locally to deploy new
applications andthen install them back. This approach is feasible but wastes
human resources and money greatly. Spontaneouslya smarter approach which can
dynamically install, run and removethe applications in the sensor networkhas
been put up. This approach, which can change the behaviorof the network
programmaticallymakes the sensor network an active network.

% ====================================================================================

\section{Solution}

% - we have a program, called Lua SPOT
%   it can run script, Lua script, on it
% - software architecture
%   - sensor network as a service provider
%   - applications are the services
%   - invoked by an RPC
%   - RPC message format
%   - API
%   - applications are equal and shared
%     - one app can call other app's function
% - Two basic applications
%   - app manager
%     - new app installation procedure

To answer the problem described in the previous section, we came up with idea
to run one or more scripts on top of the sensor network. We first run our
native application on the sensor network and provide it a basic capabilities to
run script on the application. The application handles the installation and
execution of scripts so we can dynamically install and execute scripts on
the sensor network. We call our application Lua SPOT.

As implied in the name, Lua SPOT is able to run scripts written in Lua
language. Lua is choosen because it is fairly simple yet powerful scripting
language and it is designed as an embedded scripting language. Lua needs a
virtual machine to be run on and we use Kahlua
\footnote{http://code.google.com/p/kahlua/}, an open source Lua Virtual
Machine that is written on Java langauge which can be run on top of Sun
SPOT, the sensor network that we use.

We will describe Lua SPOT in detail in the following sections.

\subsection{Software Architecture}

First of all, we would like to give a general overview of the building
blocks of Lua SPOT. As shown in Figure~\ref{fig:architecture},
generally Lua SPOT is composed of three layers including the Sun
SPOT layer where Lua SPOT runs on. This layer is where Lua SPOT gets
executed. It also provides APIs to use the wireless network, access sensors
and other input/output ports that can be used by any Sun SPOT application
that run on top of it.

\begin{figure}[htbp]
	\centering
	\scalebox{0.9}
	{
	\begin{pspicture}(0,-1.5)(9.02,1.52)
	\psframe[linewidth=0.04,dimen=outer,fillstyle=solid](9.0,1.5)(0.0,-1.5)
	\usefont{T1}{ptm}{m}{n}
	\rput(1.04,1.005){Manager}
	\rput(2.89,1.005){Router}
	\rput(4.63,1.005){App 1}
	\rput(6.32,1.005){App ...}
	\rput(8.05,1.005){App n}
	\rput(1.57,0.0050){Lua SPOT}
	\rput(6.11,0.0050){Lua VM (Kahlua)}
	\rput(4.56,-0.995){Sun SPOT}
	\psline[linewidth=0.04cm](2.0,1.5)(2.0,0.5)
	\psline[linewidth=0.04cm](3.8,1.5)(3.8,0.5)
	\psline[linewidth=0.04cm](5.4,1.5)(5.4,0.5)
	\psline[linewidth=0.04cm](7.2,1.5)(7.2,0.5)
	\psline[linewidth=0.04cm](0.0,0.5)(9.0,0.5)
	\psline[linewidth=0.04cm](0.0,-0.5)(9.0,-0.5)
	\psline[linewidth=0.04cm](3.2,0.5)(3.2,-0.5)
	\end{pspicture} 
	}
	\label{fig:architecture}
	\caption{Software Architecture}
\end{figure}

The middle layer is where Lua SPOT resides. We also put the Lua VM
there since it will be part of Lua SPOT. This layer provides a
script execution environment that handles scripts execution as well as
installation and removal. APIs exposed by the Sun SPOT layer will also be
exported to the scripts by creating function wrappers that can be called
from Lua scripts above.

% FIXME: 
% - tells about applications and they will be written in Lua (except
%   manager)
Lua scripts (or application) run on the top layer and they will be the
application logic that controls the sensor network. The scripts will be run
on the Lua virtual machine and can use APIs that are provided by the Lua
SPOT to access functions that are provided by Sun SPOT and the Lua SPOT
itself.

\subsection{Service Provider}

In Lua SPOT, we introduce a term Service Provider. We design the sensor
network as a service provider that provides services which can be used by
other entities.

Applications on the Lua SPOT will provide services. Functions inside each
application can be called by other entities using some mechanism. Since the
interaction between the functions and other entities is basically using
function call, or remote function call to be precise, therefore an RPC like
mechanism will be used to invoke a function inside an application.

Other entities that want to access a function inside a sensor network
should send a message that represents an RPC. For the sake of simplicity,
we design the RPC message as shown in Figure~\ref{fig:rpc}.

\begin{figure}[htbp]
	\centering
	\scalebox{0.8} % Change this value to rescale the drawing.
	{
	\begin{pspicture}(0,-0.42)(10.2,0.42)
	\psframe[linewidth=0.04,dimen=outer,fillstyle=solid](10.2,0.4)(0.0,-0.4)
	\usefont{T1}{ptm}{m}{n}
	\rput(0.4,-0.015){ID}
	\rput(2.0,0.015){Application}
	\rput(4.2,-0.015){Function}
	\rput(7.8,-0.0050){Parameter}
	\psline[linewidth=0.04cm](0.8,0.4)(0.8,-0.4)
	\psline[linewidth=0.04cm](3.2,0.4)(3.2,-0.4)
	\psline[linewidth=0.04cm](5.2,0.4)(5.2,-0.4)
	\end{pspicture} 
	}
	\label{fig:rpc}
	\caption{RPC Message Format}
\end{figure}

It contains four fields and each field is separated by a single space. The
first field will be the message identifier which distinguish our packet
with any other packet. It contains a single character \texttt{c}, a short
for "call".

The second and third field will be the application and function identifier.
At the moment, we simply use the application and function name. The last
field is the parameter that will be supplied to the function. Multiple
parameters will be merged into one parameter and it is the function
responsibility to parse the parameter into multiple parameters.

\subsection{Application Execution}

When the connection listener in the Lua SPOT receives a message, it will
create a new thread and pass the message to a function called
\texttt{dispatch()} from inside the thread.  This function handles the
initial message processing. It drops unwanted messages and extracts the
application and function name as well as the parameter.

After knowing the application and function name, the \texttt{dispatch()}
function will create a new Lua Virtual Machine and invoke the requested
application and function. Each new Lua Virtual Machine contains the Lua
standard library, the Sun SPOT and Lua SPOT libraries will be described
later, the requested application code, and all other installed
applications.

% TODO: elaborate more
All installed applications will be available inside the Lua Virtual
Machine. One application can call function on another application. 

Since each function call is executed under a separate Lua Virtual Machine,
this means the function call is stateless because any state will be
destroyed once the function returns. One function execution and other
function executions that happen at the same time can't interfere each
other. 

However, a special APIs are provided by the Lua SPOT that allow the
applications to share a global state. This will be described in the next
section.

\subsection{Sun SPOT and Lua SPOT APIs}



\subsection{Basic Applications}

fsfsfsd

% ====================================================================================

\section{Active Network}

% - we can deploy new application
% - router is implemented as an application
% - we can change the router behaviour programmatically
%   - by changing the routing application
%   - by changing a specific application that determine behavior
%     - ex: router calls find_next_hop
%     - we can change the find_next_hop function
%     - and therefore the routing behaviour will change accordingly
% - this fits the active network concept

% ====================================================================================

\section{Experiments}

% we build a host application that uses services on the sensor networks
% - insert new application
% - dynamic path calculation
%   - each sensor network measure a value
%   - they exchange the value
%   - they calculate the path according to the values
%   - the path will be used as a routing path

\subsection{Topology}

%\begin{figure}[htbp]
%	\centering
%	\includegraphics[width=8cm]{topology}
%	\label{fig:network-topology}
%	\caption{Network Topology}
%\end{figure}

Multiple applications will run on the host, collecting different kinds of data
from different applications in the sensor network. And new applications can be
selected from the host workstation to be broadcasted to install in the network.
Different nodes in the sensor network will communicate with each other by means
of radio connection. 

\subsection{Expectation}

We expect new Lua applications can be selected from host application and be
installed in the sensor network. The Lua application will be compiled to binary
format beforehand. And we also expect that applications can be called and
removed when received certain messages we specify in the above section. 

\subsection{Results}

In our experiments, the Lua application can really be managed dynamically in
the sensor network.  The new Lua application, which is to light on the SunSpots
one by one in the sequence of devices� orientation can be deployed in the whole
sensor network. After installing this application, a message to run this
application from the host will make all the nodes work coordinately to finish
this task. And a remove message can make all the nodes remove this application,
thus this application can no longer be called. 


% ====================================================================================

\section{Conclusion}

% - we have successfully create a scriptable sensor network
% - limitations
%   - at the moment, lua spot doesn't care about memory consumption
%   - 
% - future works
%   - handle the limitations
%   - enrich lua spot api
%   - 



\begin{thebibliography}{1}

\bibitem{wsn}
http://en.wikipedia.org/wiki/Wireless\_Sensor\_Networkas

\end{thebibliography}

\end{document}

