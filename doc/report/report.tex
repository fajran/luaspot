
% TODO:
% - fix image caption numbering

\documentclass[conference]{IEEEtran}

\usepackage{graphics}
\usepackage{graphicx}
\usepackage{pstricks}
\usepackage{epsfig}
\usepackage{pst-grad} % For gradients
\usepackage{pst-plot} % For axes

\begin{document}

\title{Scriptable Sensor Network}

\author{\IEEEauthorblockN{Fajran Iman Rusadi}
\IEEEauthorblockA{Universiteit van Amsterdam\\
Email: frusadi@science.uva.nl}
\and
\IEEEauthorblockN{ZhengZhangzheng}
\IEEEauthorblockA{Universiteit van Amsterdam\\
Email: zzheng@science.uva.nl}}

\maketitle

\begin{abstract}
Each sensor node in the sensor networks is typically programmed for a
certain task. Changing task means changing application that is run on the
sensor node and this is difficult to be done manually since the number of
sensor nodes might be very large. We provide a solution to make a
scriptable sensor network that also allows us to dynamically install new
applications through the network. We will also show that our solution can
also be used as a foundation to build an active network.
\end{abstract}


\section{introduction}

% - sensor network
%   - sun spot
%     -> we are going to use sun spot
% - scripting language
%   -> what can scripting does
%      -> dynamic execution
% - active network
%   - general overview => what is it
%   - what is the diffence between active network and regular/passive
%   network
%   - what can it does

% Sensor network
% - Tell about sensor network, what is it. Find a reference from a paper,
%   books, journal. NOT wikipedia.
% - Tell that we are going to use Sun SPOT
% Scripting language
% Acive network
% - general overview => what is it
% - what is the diffence between active network and regular/passive
% network
% - what can it does


A sensor network is a network that is composed of a collection of sensor
node \cite{journals/cn/AkyildizSSC02}, a small device that has sensors to
collect data, processor, and communication component. Sensor networks are
used for several applications like measuring temperature, humidity,
pressure, and other data that can be measured from environment.

Each sensor node in the sensor networks is typically programmed for a
certain task. Changing task means changing application that is run on the
sensor node. Application installation on the sensor network become
something that is difficult to be done \cite{MuAlKo2007}. 

To handle this problem, we propose a solution to make the sensor network
scriptable. Meaning that we can run scripts on the sensor node that also
can be installed dynamically over the network. Hence, application
installation become easier.

The application architecture that is used in our solution is flexible and
also powerful to handle many kind of application. Not only applications
that use the sensor node and the network, but also applications that
control the network. We will show this by building an active network based
on our solution. An active network is a network architecture that allows
users to inject programs to the network nodes
\cite{journals/ccr/TennenhouseW07} that later on can do some computations
on the messages flowing through them \cite{Ten97}. We call this as
programmatically change the network behavior.

% \subsection{Wireless Sensor Network}
% 
% Awireless sensor network (WSN) is a wireless network consisting of spatially
% distributed autonomousdevices using sensorsto cooperatively monitor physical or
% environmental conditions, such as temperature, sound, vibration, pressure,
% motion or pollutants, at different locations \cite{wsn} This
% resource-constrained network is usually self-configured and data centric, and
% always has dynamic topology and a specific application running on it. In our
% experiments, we use SunSpots, powerful sensor devices which are easier to
% program to make a more flexible sensor network. 
% 
% \subsection{Scripting Language}
% 
% A scripting languageis a programming languagethat allows some control of a
% single or many software applications. Scripts are often interpreted from the
% source code or "semi-compiled" to bytecodewhich is interpreted, unlike the
% applications they are associated with, which are traditionally compiledto
% native machine code for the system on which they run. Scripting languages are
% nearly always embedded in the application with which they are associated. With
% thischaracteristicof scriptinglanguage, it can support dynamic execution fairly
% well. 
% 
% \subsection{Concept of Active Network}
% 
% Active networkis an architectural framework to allow extension ofnetwork
% services by users. It can support application-specific network (layer)
% servicesand programs can inject code fragments to decide how their traffic
% isprocessed by the network.The nodes of active network are programmed to
% perform custom operations on the messages that pass through the node. For
% example, a node could be programmed or customized to handle packets on an
% individual user basis or to handle multicastpackets differently than other
% packets. 

% ====================================================================================

\section{Background}

% - problems:
%   - we want to utilize h/w resource
%   - we want to run different applications
%     - so we have to put the prog on the sensor network
%     - but if we always have to do this manually, we will
%       have difficulties in doing it: 
%       - waste time, human resources, money, etc
%   - so, we want to be able to deploy/install/put the app
%     dynamically (remotely) to all sensor networks
%   - we also want to change the network behaviour programmatically (active network)

% At present, most state of the art sensor networksare designed for a single
% application. We can bear this approach on the condition that the device in the
% sensor network is not so powerful and expensive. But with the advent of more
% powerful senor devices, we want to utilize the hardware we invest in the sensor
% network to a bigger extent, hoping that it can run multiple applications in it,
% not restricted by simply measuring temperatureand humidity. Traditionally, the
% administrator of the sensor network can collect all the devices in the network
% from different locations, upgrade the firmware locally to deploy new
% applications andthen install them back. This approach is feasible but wastes
% human resources and money greatly. Spontaneouslya smarter approach which can
% dynamically install, run and removethe applications in the sensor networkhas
% been put up. This approach, which can change the behaviorof the network
% programmaticallymakes the sensor network an active network.


% - sensor network is used to do a certain task that is done by a certain
%   application
% - task may changes, therefore we need to install new application that
%   does the new task
% - installing new app might be not an easy job. visiting each
%   location one by one wastes human resources and money greatly.
% - the cost higher as the number of sensor networks increases
% - beside deploying new app to do another task, we might also need to
%   change the way the network works
%   - for example, we need to make a better routing algorithm that
%     pays attention to the battery level
%   - need to make segmentation in the network
%   - this leads to issue of creating active network

Normally, a sensor network is used to do a certain task such as doing
temperature measurement. This task is usually done by an application
installed on each sensor node. When we want to make the sensor network
do another task, we need to install a new application that does the new
task.

Installing new application might be not an easy task \cite{MuAlKo2007}.
Having to visit each sensor node and install new application one by one
requires a lot of human resources and costs money. This cost become higher
as more sensor networks deployed in the field.

To reduce this cost, we need some mechanism that allows us to dynamically
install new applications on the sensor nodes. Dynamic here means we can
install or remove applications over the network itself and therefore we
don't have to individually visit the sensor node and install the new
application.

Beside that, we might also need to change the network behavior. For
example, to save more battery power, we need to use a better routing
algorithm that pays attention to the battery power level. Another example
is when we need to make segmentation in the network. We have to make sure
that a message in one network segment is not broadcasted or forwarded to
the another network segment.

This idea of changing network behavior leads to issue of creating active
network. We want to control the network behavior programmatically by
sending messages that can change the way the network works.


% ====================================================================================

\section{Lua SPOT}

% - we have a program, called Lua SPOT
%   it can run script, Lua script, on it
% - software architecture
%   - sensor network as a service provider
%   - applications are the services
%   - invoked by an RPC
%   - RPC message format
%   - API
%   - applications are equal and shared
%     - one app can call other app's function
% - Two basic applications
%   - app manager
%     - new app installation procedure

To answer the problem described in the previous section, we came up with idea
to run one or more scripts on top of the sensor node. We first run our
native application on the sensor node and provide it a basic capabilities to
run script on the application. The application handles the installation and
execution of scripts so we can dynamically install and execute scripts on
the sensor node. We call our application Lua SPOT.

As implied in the name, Lua SPOT is able to run scripts written in Lua
language. Lua is choosen because it is fairly simple yet powerful scripting
language and it is designed as an embedded scripting language
\cite{Ieru96a}. Lua needs a virtual machine to be run on and we use Kahlua
\footnote{http://code.google.com/p/kahlua/}, an open source Lua Virtual
Machine that is written on Java langauge which can be run on top of Sun
SPOT, the sensor node that we use.

To avoid confusion, we will use term \textbf{Lua SPOT} if we refer to the
native application that we build on Sun SPOT. The Lua scripts will be
addressed as \textbf{applications} that run on the Lua SPOT.

We will describe Lua SPOT in more detail in the following sections.

\subsection{Software Architecture}

First of all, we would like to give a general overview of the building
blocks of Lua SPOT. As shown in Figure~\ref{fig:architecture},
generally Lua SPOT is composed of three layers, including the Sun
SPOT layer where Lua SPOT gets executed. It also provides APIs to use the
wireless network, access sensors and other input/output ports that can be
used by any Sun SPOT application that run on top of it.

\begin{figure}[htbp]
	\centering
	\scalebox{0.9}
	{
	\begin{pspicture}(0,-1.5)(9.02,1.52)
	\psframe[linewidth=0.04,dimen=outer,fillstyle=solid](9.0,1.5)(0.0,-1.5)
	\usefont{T1}{ptm}{m}{n}
	\rput(1.04,1.005){Manager}
	\rput(2.89,1.005){Router}
	\rput(4.63,1.005){App 1}
	\rput(6.32,1.005){App ...}
	\rput(8.05,1.005){App n}
	\rput(1.57,0.0050){Lua SPOT}
	\rput(6.11,0.0050){Lua VM (Kahlua)}
	\rput(4.56,-0.995){Sun SPOT}
	\psline[linewidth=0.04cm](2.0,1.5)(2.0,0.5)
	\psline[linewidth=0.04cm](3.8,1.5)(3.8,0.5)
	\psline[linewidth=0.04cm](5.4,1.5)(5.4,0.5)
	\psline[linewidth=0.04cm](7.2,1.5)(7.2,0.5)
	\psline[linewidth=0.04cm](0.0,0.5)(9.0,0.5)
	\psline[linewidth=0.04cm](0.0,-0.5)(9.0,-0.5)
	\psline[linewidth=0.04cm](3.2,0.5)(3.2,-0.5)
	\end{pspicture} 
	}
	\label{fig:architecture}
	\caption{Software Architecture}
\end{figure}

The middle layer is where Lua SPOT resides. We also put the Lua VM
there since it will be part of Lua SPOT. This layer provides a
script execution environment that handles scripts execution as well as
installation and removal. APIs exposed by the Sun SPOT layer will also be
exported to the scripts by creating function wrappers that can be called
from Lua scripts above.

% FIXME: 
% - tells about applications and they will be written in Lua (except
%   manager)
Lua scripts run at the top layer and they will be the application logic that
controls the sensor node. The scripts will be run on the Lua virtual machine
and can use APIs that are provided by the Lua SPOT to access functions that are
provided by Sun SPOT and the Lua SPOT itself.

\subsection{Service Provider}

In Lua SPOT, we introduce a term Service Provider. We design the sensor
network as a service provider that provides services which can be used by
other entities.

Applications on the Lua SPOT provide services. Functions inside each
application can be called by other entities using some mechanism. Since the
interaction between the functions and other entities is basically using
function call, or remote function call to be precise, therefore an RPC like
mechanism will be used to invoke a function inside an application.

Other entities that want to access a function inside a sensor node
should send a message that represents an RPC. For the sake of simplicity,
we design the RPC message as shown in Figure~\ref{fig:rpc}.

\begin{figure}[htbp]
	\centering
	\scalebox{0.8} % Change this value to rescale the drawing.
	{
	\begin{pspicture}(0,-0.42)(10.2,0.42)
	\psframe[linewidth=0.04,dimen=outer,fillstyle=solid](10.2,0.4)(0.0,-0.4)
	\usefont{T1}{ptm}{m}{n}
	\rput(0.4,-0.015){ID}
	\rput(2.0,0.015){Application}
	\rput(4.2,-0.015){Function}
	\rput(7.8,-0.0050){Parameter}
	\psline[linewidth=0.04cm](0.8,0.4)(0.8,-0.4)
	\psline[linewidth=0.04cm](3.2,0.4)(3.2,-0.4)
	\psline[linewidth=0.04cm](5.2,0.4)(5.2,-0.4)
	\end{pspicture} 
	}
	\label{fig:rpc}
	\caption{RPC Message Format}
\end{figure}

It contains four fields and each field is separated by a single space. The
first field will be the message identifier which distinguish our packet
with any other packets. It contains a single character \texttt{c}, a short
for "call".

The second and third field will be the application and function identifier.
At the moment, we simply use the application and function name. The last
field is the parameter that will be passed to the function. Multiple
parameters will be merged into one parameter and it is the function
responsibility to parse the parameter into multiple parameters.

Based on this very basic idea of making the sensor network as a service
provider, we can build many useful and powerful applications. For example,
a routing function can be implemented as just another function. Messages
that need to be routed will be inserted as an embedded message of a
function call to a routing function. In the last part of this section, we
will show you an example of routing function that we make as part of basic
pre-installed applications.

If we eventually need a better routing function, we can replace the
existing routing function or even create a new function. So, changing the
behaviour of a network can be as simple as inserting a new application.

\subsection{Application Execution}

When the connection listener in the Lua SPOT receives a message, it will
create a new thread and pass the message to a function called
\texttt{dispatch()} from inside the thread. This function handles the
initial message processing. It drops unwanted messages and extracts the
application and function name as well as the parameter.

After knowing the application and function name, the \texttt{dispatch()}
function will create a new Lua Virtual Machine and invoke the requested
application and function. Each new Lua Virtual Machine contains the Lua
standard library, the Sun SPOT and Lua SPOT libraries will be described
later, the requested application code, and all other installed
applications.

% TODO: elaborate more
All installed applications will be available inside the Lua Virtual
Machine. One application can call function on another application.

Since each function call is executed under a separate and new Lua Virtual
Machine, this means the function call is stateless because any state will
be destroyed once the function returns. In addition, multiple function
executions that happen at the same time can't interfere each other. 

However, a special APIs are provided by the Lua SPOT that allow the
applications to share a global state. This will be described in the next
section.

\subsection{Sun SPOT and Lua SPOT APIs}

Originally, the APIs that are available inside the Lua Virtual Machine is
very limited. It only contains the Lua standard library and this makes the
scripts run on it are practically useless since they can only process
something (the parameter) from the network but can't do something to the
network.

Therefore, a set of functions, that are grouped into two API sets, are
provided: Sun SPOT and Lua SPOT APIs. The Sun SPOT API contains functions
to access facilities provided by the Sun SPOT, such as sensors and other
input/output ports. The second API contains other functions that are
required to make applications run on Lua SPOT more powerful, such as the
global memory storage, synchronization, and also function to send message
to the network.

\subsection{Basic Applications}

There are two basic applications in the Lua SPOT: Application Manager and
Basic Router. The first application is responsible for adding, removing,
and installing new application. The second application provides a routing
function to route a message from one sensor node to other sensor nodes.

\subsubsection{Application Manager}

Installing a new application in Sun SPOT is also just a matter of calling a
function. A default application, called Application Manager, is reponsible to
handle this kind of things. This application, which is named
\texttt{manager}, is written in Java and will be the only application that
is written in Java.

The application has two main functions: \texttt{install} and
\texttt{remove}. The \texttt{remove} function need an application name that
will be removed as the parameter. The \texttt{install} function needs more
complex structure of the parameter since special care is needed when
receiving new application data.

The size of message that can be transmitted to the network in one go is
limited. This can be imagined as the MTU in the regular network.  Since the
application size can be larger than the MTU, a data fragmentation is
needed. Therefore, the parameter of the \texttt{install} function contains
information about fragments. Figure~\ref{fig:install} shows the message
structure that is expected by the \texttt{install} function.

\begin{figure}[htbp]
	\centering
	\scalebox{0.8} % Change this value to rescale the drawing.
	{
	\begin{pspicture}(0,-0.9)(10.22,0.92)
	\psframe[linewidth=0.04,dimen=outer,fillstyle=solid](10.2,0.9)(0.0,-0.9)
	\usefont{T1}{pcr}{m}{n}
	\rput(0.38,0.435){c}
	\rput(1.9,0.495){manager}
	\rput(3.98,0.435){install}
	\usefont{T1}{ptr}{m}{n}
	\rput(5.7,0.435){name}
	\rput(7.1,0.435){index}
	\rput(8.9,0.435){fragments}
	\rput(5.1,-0.385){application data}
	\psline[linewidth=0.04cm](0.0,0.1)(10.2,0.1)
	\psline[linewidth=0.04cm](0.8,0.9)(0.8,0.1)
	\psline[linewidth=0.04cm](3.0,0.9)(3.0,0.1)
	\psline[linewidth=0.04cm](5.0,0.9)(5.0,0.1)
	\psline[linewidth=0.04cm](6.4,0.9)(6.4,0.1)
	\psline[linewidth=0.04cm](7.8,0.9)(7.8,0.1)
	\end{pspicture} 
	}
	\label{fig:install}
	\caption{Application Installation Packet Structure}
\end{figure}

The function requires four parameters: the application name, the fragment
index, the total fragments, and the application data. When receiving new
install request (identified by index 0), the application manager will
initialize new application slot in the memory. The application data is
appended to that slot until the last fragment arrives. The application
manager will then activate the application so it is ready to be used.

\subsubsection{Basic Router}

When a sensor node want to send a message to another sensor node, there
is a case when they are not within a range so the message should be
transmitted through one or more intermediate sensor nodes. In order to
do this, a special function need to be used so an intermediate sensor
network can receive and forward such message. This functionality is called
routing.

Lua SPOT contains an application that does routing. It receives a message
and if the message is not sent to it, it will forward the message. The
message will be hopped from one sensor node to the other until
eventually it reaches the destination. If the routing application receives
a message that is directed to it, it will remove the routing header and
process the original message. The processing is done by calling the
\texttt{dispatch()} function from inside application.

This routing application might be very basic, but it can already be used to
forward a message from one node to another node that are not withing a
range. This application is implemented as a Lua script and can be replaced
if needed.

Just like the install function, routing function is implemented as another
function that can be called. The name of application is \texttt{routing}
and the function that will do routing is \texttt{route}. The function
expects several parameters as shown in Figure~\ref{fig:routing}.

\begin{figure}[htbp]
	\centering
	\scalebox{0.9} % Change this value to rescale the drawing.
	{
	\begin{pspicture}(0,-1.3)(8.62,1.32)
	\psframe[linewidth=0.04,dimen=outer,fillstyle=solid](8.6,1.3)(0.0,-1.3)
	\usefont{T1}{pcr}{m}{n}
	\rput(0.38,0.835){c}
	\rput(1.78,0.835){router}
	\rput(3.56,0.835){route}
	\usefont{T1}{ptr}{m}{n}
	\rput(5.4,0.835){msg id}
	\rput(7.48,0.835){maxhop}
	\rput(2.38,0.0050){src addr}
	\rput(6.38,0.0050){dst addr}
	\rput(4.47,-0.785){nested message}
	\psline[linewidth=0.04cm](0.0,0.5)(8.6,0.5)
	\psline[linewidth=0.04cm](0.8,1.3)(0.8,0.5)
	\psline[linewidth=0.04cm](2.8,1.3)(2.8,0.5)
	\psline[linewidth=0.04cm](4.4,1.3)(4.4,0.5)
	\psline[linewidth=0.04cm](6.4,1.3)(6.4,0.5)
	\psline[linewidth=0.04cm](0.0,-0.3)(8.6,-0.3)
	\psline[linewidth=0.04cm](4.4,0.5)(4.4,-0.3)
	\end{pspicture} 
	}
	\label{fig:routing}
	\caption{Router Packet Structure}
\end{figure}

The message contains message id that will be used to avoid processing
duplicate messages, it also has a maximum hop number to limit the
message distribution. Source and destrination addresses are obviusly used
to state the sender and receiver. The last part of the message is another
message that is embedded in this message. As mentioned earlier, that nested
message will be sent to the \texttt{dispatch()} function so it can be
processed just like any other message.


% ====================================================================================

\section{Lua SPOT and Active Network}

% - we can deploy new application
% - router is implemented as an application
% - we can change the router behaviour programmatically
%   - by changing the routing application
%   - by changing a specific application that determine behavior
%     - ex: router calls find_next_hop
%     - we can change the find_next_hop function
%     - and therefore the routing behaviour will change accordingly
% - this fits the active network concept

One of the key concept of Active Network is an ability to change network
behavior programmatically \cite{journals/ccr/TennenhouseW07}. A message
that is sent to a node may contain a program that will be executed on the
node. This program will then run and control the traffic that passes
through the node. Furthermore, a message might be an active "capsule", a
small program that gets executed in every router/switch that it traverses.

In this section, we would like to show that Lua SPOT can be used as a
foundation in building an active network. We will try to design a Lua SPOT
application that can show features of active network.

\subsection{Interacting with Lua SPOT}

The basic and only interaction with Lua SPOT is by sending an RPC message.
Lua SPOT does nothing but accepting an RPC message and dispatch the message
to the corresponding application and function. Therefore, all features that
are expected from a network device are implemented as functions inside
applications or, as we mention earlier, services.

Lua SPOT has a very main function that acts as gateway of command
execution, the \texttt{dispatch()} function. This function is called when
Lua SPOT receives a new message. This function can also be called from
inside an application which opens possibility of executing any message that
is created from inside the application, including a message that is taken
from the parameter of the function. That is, a message that is passed as a
parameter through the \texttt{dispatch()} function that might come
from other network elements.

At the current implementation of Lua SPOT, the created Lua Virtual Machines
will contains all installed applications. A function that is invoked can
access all other available functions. In this way, any new application that
is installed in the sensor node will enrich the functionality of the
sensor node, and therefore the sensor network. The installed application
can have functions that are not intended to be called remotely, but solely
for making the sensor node richer.

\subsection{Creating Active Network}

Based on Lua SPOT approach described in the previous section, we will
design an application that can support two scenarios that are expected from
an Active Network, as discussed in \cite{journals/ccr/TennenhouseW07}. An
ability to execute a program embedded in the message and let the program to
control the traffic; and a support to send "capsule" to the network, a
message that contains an embedded program which will be executed on the
receiving nodes and forwarded to other nodes.

The first scenario can be achieved at least in two ways. The first one is
by replacing the router application by another application that is smarter
in processing the incoming messages. If this routing function is a standard
routing function that is used to carry all other messages, replacing it
means changing the way the network works.

Another way to achieve the first scenario is by inserting a new routing
function. Any other messages that we want to transmit can be carried using
that routing function. Therefore, if we have messages that need to be
handled (routed) differently, we can use this new routing function to carry
them.

A "capsule" message can be handled by Lua SPOT by having a function that
does two things: routing and dispatching. When the function receives a
message, it will execute the embedded message by passing it to the
\texttt{dispatch()} function and also forward the message to the other
nodes. In this way, a message will get executed in every intermediate nodes
that it traverses.


% ====================================================================================

\section{Experiments}

% we build a host application that uses services on the sensor networks
% - insert new application
% - dynamic path calculation => we want to get a result from a collective work of sensor networks
%   - each sensor network measure a value
%   - they exchange the value
%   - they calculate the path according to the values
%   - the path will be used as a routing path

% Tell that we had done two experiments
% - application installation
% - collective behavior test
% Tell about our network configuration (topology)
% - we have a base station
% - we have a group of sensor networks (3)
% First experiment
% - we prepare a Lua script and then compile it
% - we send the script to all sensor networks from the base station
% - and we know we have sucessfully installed the app by calling
%   the new function
% Second experiment
% - we are trying to setup routing path in the sensor network
% - we don't manually create the path by telling each sensor network
% - we send a command to the group and let them to:
%   - exchange informations
%   - use those informations to calculate the path
% - we send a command to show the path

To test Lua SPOT, we made two experiments. The first experiment is about
testing the installation function. If this experiment is successful, then
we can answer the problem described in the earlier section: we need to be
able to dynamically deploy new applications on the sensor nodes.

In the second experiment, we would like to show a collective behavior of
sensor nodes. We inserted an application that will exchange a
measurement data to other sensor nodes. Then, each sensor node will
independently calculate the data to build a routing path. Another command
later on will be sent to the sensor network and they should be able to
pass the command one by one along the path they made earlier.

% \subsection{Network Configuration}

\subsection{Application Installation}

In this experiment, we tried to send a installation command message. We
installed several applications ranging from small application to a
relatively large application, in terms of file size.

We played around with the fragment size that still can be used to send each
fragment successfully. Based on our experience, a relatively small
fragment size still has high error possibility when the network is
relatively busy.

\subsection{Collective Behaviour of Sensor Nodes}

This experiment has a goal to show a collective behaviour of sensor
nodes in the network. We were trying to make a routing path that is
determined by a certain measurement, we use the value of accelerometer
provided by Sun SPOT, taken by each sensor node. 

The idea was to make each sensor node exchange the value to all other
sensor nodes. Then each sensor node individually calculate a routing path
using the values. In our experiment, we simply make the sensor node to sort
the values in order to make routing path. The routing path calculation is
done in a distributed fashion since there is no single sensor node that
acts as a central controller. However, we still need a sensor node that
will trigger the value exchange by sending a message. 

After that, we send a message to all sensor networks that should
be responded by a sensor node that has a smallest measurement value. It
then send a message to its next sensor node in the routing path calculated
earlier.

% \subsection{Topology}
% 
% %\begin{figure}[htbp]
% %	\centering
% %	\includegraphics[width=8cm]{topology}
% %	\label{fig:network-topology}
% %	\caption{Network Topology}
% %\end{figure}
% 
% Multiple applications will run on the host, collecting different kinds of data
% from different applications in the sensor network. And new applications can be
% selected from the host workstation to be broadcasted to install in the network.
% Different nodes in the sensor network will communicate with each other by means
% of radio connection. 
% 
% \subsection{Expectation}
% 
% We expect new Lua applications can be selected from host application and be
% installed in the sensor network. The Lua application will be compiled to binary
% format beforehand. And we also expect that applications can be called and
% removed when received certain messages we specify in the above section. 
% 
% \subsection{Results}
% 
% In our experiments, the Lua application can really be managed dynamically in
% the sensor network.  The new Lua application, which is to light on the SunSpots
% one by one in the sequence of devices� orientation can be deployed in the whole
% sensor network. After installing this application, a message to run this
% application from the host will make all the nodes work coordinately to finish
% this task. And a remove message can make all the nodes remove this application,
% thus this application can no longer be called. 
% 

% ====================================================================================

\section{Conclusion}

% - we have successfully create a scriptable sensor network
% - limitations
%   - at the moment, lua spot doesn't care about memory consumption
%   - 
% - future works
%   - handle the limitations
%   - enrich lua spot api
%   - 

In conclusion, we have successfully created an application on sensor
network that is able to run scripts. Beside running the pre-installed
scripts, new scripts can also be installed in this application. This
opens posibility to add and run new scripts after the deployment of sensor
networks without having to visit each sensor node individually.

\subsection{Future Works}

The current implementation of Lua SPOT still doesn't export all functions
of Sun SPOT to the Lua VM. It also doesn't pay attention to the memory
usage of Lua VM as well as the applications that will be run on it. The
message format is also not very efficient and not really binary message
friendly. 

In addition to fix the limitation at the current implementation, what we
would like to see on Lua SPOT is some new improvements. The
\texttt{dispatch()} function currently work synhronously. Means it will
block the execution when it is called. An asynrhonous version of it might
be makes the Lua SPOT more powerfull and can do something like invoking a
function several times in relatively parallel.

To make script development easier, an SDK and simulator for Lua script
might also a good idea. The developers doesn't need Sun SPOT if they have
something that can be used to test the scripts before actually deploying
them to the Lua SPOT.

\bibliographystyle{plain}
\bibliography{report}

\end{document}

