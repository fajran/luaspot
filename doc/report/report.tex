
\documentclass[conference]{IEEEtran}

\begin{document}

\title{Scriptable Sensor Network}

\author{\IEEEauthorblockN{Fajran Iman Rusadi}
\IEEEauthorblockA{Universiteit van Amsterdam\\
Email: frusadi@science.uva.nl}
\and
\IEEEauthorblockN{ZhengZhangzheng}
\IEEEauthorblockA{Universiteit van Amsterdam\\
Email: zzheng@science.uva.nl}}

\maketitle

\begin{abstract}
Abstract 
\end{abstract}



\section{Introduction}

% - sensor network
%   - Sun SPOT
%     -> we are going to use Sun SPOT
% - scripting language
%   -> what can scripting does
%      -> dynamic execution
% - concept of active network

\section{Background}

% - problems:
%   - we want to utilize h/w resource
%   - we want to run different applications
%     - so we have to put the prog on the sensor network
%     - but if we always have to do this manually, we will
%       have difficulties in doing it: 
%       - waste time, human resources, money, etc
%   - so, we want to be able to deploy/install/put the app
%     dynamically (remotely) to all sensor networks
%   - we also want to change the network behaviour programmatically (active network)

\section{Solution}

% - we have a program, called Lua SPOT
%   it can run script, Lua script, on it
% - software architecture
%   - sensor network as a service provider
%   - applications are the services
%   - invoked by an RPC
%   - RPC message format
%   - API
%   - applications are equal and shared
%     - one app can call other app's function
% - Two basic applications
%   - app manager
%     - new app installation procedure


\section{Active Network}

% - we can deploy new application
% - router is implemented as an application
% - we can change the router behaviour programmatically
%   - by changing the routing application
%   - by changing a specific application that determine behavior
%     - ex: router calls find_next_hop
%     - we can change the find_next_hop function
%     - and therefore the routing behaviour will change accordingly
% - this fits the active network concept

\section{Experiments}

% we build a host application that uses services on the sensor networks
% - insert new application
% - dynamic path calculation
%   - each sensor network measure a value
%   - they exchange the value
%   - they calculate the path according to the values
%   - the path will be used as a routing path

\section{Conclusion}

% - we have successfully create a scriptable sensor network
% - limitations
%   - at the moment, lua spot doesn't care about memory consumption
%   - 
% - future works
%   - handle the limitations
%   - enrich lua spot api
%   - 



\begin{thebibliography}{1}

\bibitem{IEEEhowto:kopka}
H.~Kopka and P.~W. Daly, \emph{A Guide to \LaTeX}, 3rd~ed.\hskip 1em plus
  0.5em minus 0.4em\relax Harlow, England: Addison-Wesley, 1999.

\end{thebibliography}

\end{document}

